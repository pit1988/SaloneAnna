\documentclass[12pt,a4paper,titlepage]{article}
%va sempre messo article per "program documentation"
\usepackage[T1]{fontenc}
\usepackage[utf8]{inputenc}
\usepackage[T1]{fontenc}
\usepackage{titlesec}
\usepackage{hyperref}
\usepackage[a4paper,top=2cm,bottom=2cm,left=1cm,right=1cm]{geometry}
\usepackage{soulutf8,color}
\usepackage{graphicx}

\usepackage{emptypage}                     
% pagine vuote senza testatina e piede di pagina

%\usepackage{hyperref}                     
% collegamenti ipertestuali

\usepackage{fancyhdr}
% pacchetto per intestazione e pie pagina

\pagestyle{fancy}


% \\ indica interruzione di riga

% compilate 2 volte per documenti con indice

% {\em qui testo in corsivo}
% {\bfseries qui testo in grossetto}

%LISTE NUMERATE
%\begin{enumerate}
%\item primo
%\item secondo
%\item terzo
%\end{enumerate}

%LISTE PUNTATE
%\begin{itemize}
%\item primo
%\item secondo
%\item \dots
%\end{itemize}

%TABELLA
%\begin{tabular}{|c|c|c|}
%indica una tabella con 3 colonne e pos. testo centrale. La barra verticale ( | ) indica che vi e' una linea divisoria verticale tra le celle.
%\hline	linea separatrice orizzontale
%testo1& testo2& testo3\\
% & segna la fine del testo nella cella , \\ indica il fine della riga 

%GRAFICI
%\begin{figure}
%\includegraphics{filegrafico}
%comando per includere le immagini (controllare i formati)
%\caption{didascalia}
%\label{nome}
%\end{figure}


%----------------------------------------------------------- INIZIO TEMPLATE TITOLO

\usepackage{xcolor} % Importa i colori per la prima pagina
\usepackage{fix-cm} % Permette l'incremento del font oltre misura


\newcommand{\HRule}[1]{\hfill \rule{0.2\linewidth}{#1}} % Horizontal rule at the bottom of the page, adjust width here

\definecolor{grey}{rgb}{0.9,0.9,0.9} % Colore del box del titolo

\begin{document}
	
	\thispagestyle{empty} % Toglie il numero della pagina nella prima pagina
	
	%----------------------------------------------------------------------------------------
	%	TITLE SECTION
	%----------------------------------------------------------------------------------------
	
	\colorbox{grey}{
		\parbox[t]{1.0\linewidth}{
			\centering \fontsize{50pt}{80pt}\selectfont % Il primo è la grandezza del font, il secondo lo spazio lasciato
			\vspace*{0.7cm} % Spazio dall'inizio del box al titolo
			
			\raggedleft
			\includegraphics[width=0.7\linewidth]{../public_html/img/logo.png}
			
			\hfill Relazione Salone Anna \\
			
			\vspace*{0.7cm} % Spazio dalla fine del testo alla fine del box
		}
	}
	
	%----------------------------------------------------------------------------------------
	
	\vfill % Spazio dalla fine del box alle altre informazioni
	
	%----------------------------------------------------------------------------------------
	%	Informazioni sul documento
	%----------------------------------------------------------------------------------------
	
	{\centering \large 
		\hfill \textbf{Nome Progetto} Salone Anna \\
		\hfill \textbf{Redazione} 	Pietro Gabelli \\
		\hfill 						Sebastiano Marchesini \\
		\hfill						Andrea Grendene \\
		\hfill \textbf{Destinato} 	Ombretta Gaggi \\ 
		
		\HRule{1pt}
		
		\textbf{Sommario} \\
		Relazione progetto Salone Anna destinato a Tecnologie Web.
		
	} % Linea orizzontale di estetica
	
	
	%----------------------------------------------------------------------------------------
	
	\clearpage % Parta bianca finale della pagina
	
	%----------------------------------------------------------- FINE TEMPLATE TITOLO
	
	\lhead{\includegraphics[width=0.2\linewidth]{../public_html/img/logo.png}}
	\chead{}
	\lfoot{Relazione Progetto}
	\cfoot{}
	\rfoot{\thepage}
	\renewcommand{\headrulewidth}{0.2pt}
	\renewcommand{\footrulewidth}{0.2pt}
	
	
	\tableofcontents
	%crea indice automaticamente
	\thispagestyle{empty}

\newpage
%\pagenumbering{arabic}
\tableofcontents
\listoftables
\listoffigures
\newpage
\input{\docs Introduzione.tex}
\newpage
\input{\docs DescrizioneGenerale.tex}
\newpage
\input{\docs SuddivisioneLavori.tex}
\newpage
\input{\docs LinguaggiUtilizzati.tex}
\newpage
\input{\docs Architettura.tex}
\newpage   
\input{\docs Struttura.tex}
\newpage
\input{\docs Presentazione.tex}
\newpage
\input{\docs Comportamento.tex}
\newpage
\input{\docs Accessibilita.tex}
\newpage
\input{\docs GestioneDati.tex}
\newpage
\input{\docs PHP.tex}
\newpage
\input{\docs VerificaTest.tex}
%\newpage
%\appendix

%\bibliography{../bibliografia}

\end{document}
