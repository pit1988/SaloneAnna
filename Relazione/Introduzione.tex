\documentclass[12pt,a4paper,titlepage]{article}
%va sempre messo article per "program documentation"
\usepackage[italian]{babel}
\usepackage[T1]{fontenc}
\usepackage[latin1]{inputenc}
\usepackage{titlesec}
\usepackage{hyperref}
\usepackage[a4paper,top=2cm,bottom=2cm,left=1cm,right=1cm]{geometry}
\usepackage{soulutf8,color}
\usepackage{graphicx}

\usepackage{emptypage}                     
% pagine vuote senza testatina e piede di pagina

%\usepackage{hyperref}                     
% collegamenti ipertestuali

\usepackage{fancyhdr}
% pacchetto per intestazione e pie pagina

\pagestyle{fancy}


% \\ indica interruzione di riga

% compilate 2 volte per documenti con indice

% {\em qui testo in corsivo}
% {\bfseries qui testo in grossetto}

%LISTE NUMERATE
%\begin{enumerate}
%\item primo
%\item secondo
%\item terzo
%\end{enumerate}

%LISTE PUNTATE
%\begin{itemize}
%\item primo
%\item secondo
%\item \dots
%\end{itemize}

%TABELLA
%\begin{tabular}{|c|c|c|}
%indica una tabella con 3 colonne e pos. testo centrale. La barra verticale ( | ) indica che vi e' una linea divisoria verticale tra le celle.
%\hline	linea separatrice orizzontale
%testo1& testo2& testo3\\
% & segna la fine del testo nella cella , \\ indica il fine della riga 

%GRAFICI
%\begin{figure}
%\includegraphics{filegrafico}
%comando per includere le immagini (controllare i formati)
%\caption{didascalia}
%\label{nome}
%\end{figure}


%----------------------------------------------------------- INIZIO TEMPLATE TITOLO

\usepackage{xcolor} % Importa i colori per la prima pagina
\usepackage{fix-cm} % Permette l'incremento del font oltre misura


\newcommand{\HRule}[1]{\hfill \rule{0.2\linewidth}{#1}} % Horizontal rule at the bottom of the page, adjust width here

\definecolor{grey}{rgb}{0.9,0.9,0.9} % Colore del box del titolo

\begin{document}
	
	\thispagestyle{empty} % Toglie il numero della pagina nella prima pagina
	
	%----------------------------------------------------------------------------------------
	%	TITLE SECTION
	%----------------------------------------------------------------------------------------
	
	\colorbox{grey}{
		\parbox[t]{1.0\linewidth}{
			\centering \fontsize{50pt}{80pt}\selectfont % Il primo è la grandezza del font, il secondo lo spazio lasciato
			\vspace*{0.7cm} % Spazio dall'inizio del box al titolo
			
			\raggedleft
			\includegraphics[width=0.7\linewidth]{../public_html/img/logo.png}
			
			\hfill Relazione Salone Anna \\
			
			\vspace*{0.7cm} % Spazio dalla fine del testo alla fine del box
		}
	}
	
	%----------------------------------------------------------------------------------------
	
	\vfill % Spazio dalla fine del box alle altre informazioni
	
	%----------------------------------------------------------------------------------------
	%	Informazioni sul documento
	%----------------------------------------------------------------------------------------
	
	{\centering \large 
		\hfill \textbf{Nome Progetto} Salone Anna \\
		\hfill \textbf{Redazione} 	Pietro Gabelli \\
		\hfill 						Sebastiano Marchesini \\
		\hfill						Andrea Grendene \\
		\hfill \textbf{Destinato} 	Ombretta Gaggi \\ 
		
		\HRule{1pt}
		
		\textbf{Sommario} \\
		Relazione progetto Salone Anna destinato a Tecnologie Web.
		
	} % Linea orizzontale di estetica
	
	
	%----------------------------------------------------------------------------------------
	
	\clearpage % Parta bianca finale della pagina
	
%----------------------------------------------------------- FINE TEMPLATE TITOLO

\lhead{\includegraphics[width=0.2\linewidth]{../public_html/img/logo.png}}
\chead{}
\lfoot{Relazione Progetto}
\cfoot{}
\rfoot{\thepage}
\renewcommand{\headrulewidth}{0.2pt}
\renewcommand{\footrulewidth}{0.2pt}


\tableofcontents
%crea indice automaticamente
\thispagestyle{empty}

\newpage


\rhead{Introduzione}
\section{Introduzione}
	\subsection{Descrizione generale}
	Il progetto SaloneAnna � strutturato su una base di dati mysql per la gestione di un negozio di parrucchieri. Fulcro del programma risiede nell'amministrazione dei clienti, salvati nel database e collegati agli appuntamenti gestiti dall'amministratore.\\
	Il contesto su cui si appoggia la scelta di questo programma riguarda un salone di piccole dimensioni, si parla di un prodotto accessibile da tutte le dipendenti e titolari per tenere aggiornato il negozio con la massima efficienza.\\
	Inoltre l'amministratore del salone e chi possiede l'accesso al sito hanno la possibilit� di gestire un magazzino, cancellando, inserendo o modificando ogni singolo prodotto, in modo da avere un inventario in evoluzione e ben gestibile. Un utente non registrato invece potr� visualizzare le informazioni generiche del negozio, sfogliare una galleria dinamica delle realizzazioni e contattare tramite form il negozio.\\
	\subsection{Caratteristiche degli utenti}{
		Gli utenti del sito saranno persone in ricerca di un nuovo negozio nella zona, in grado di utilizzare gli strumenti per la navigazione web o alle prime armi. \\
		Il sito � rivolto ad un pubblico generico, all'interno del quale possiamo individuare le seguenti categorie:
		\begin{description}\itemsep1pt
			\item[Categoria di utenti:] privati;
			\begin{description}\itemsep1pt
				\item[Funzionalit�:] Informarsi sulla locazione e sul numero del salone. Consultare realizzazioni e contattare il negozio lasciando il proprio indirizzo email.
				\item[Termini generali:] Non eccessivamente distante dal punto vendita, in un raggio di circa 70 Km.
			\end{description}
			\item[Categoria di utenti:] amministratori;
			\begin{description}\itemsep1pt
				\item[Funzionalit�:] area riservata in cui poter aggiungere, rimuovere o aggiornare i prodotti, inserire nuovi clienti e conseguentemente gestire gli appuntamenti. 
			\end{description}
		\end{description}
	\subsection{Vincoli generali}{
		\begin{itemize}\itemsep1pt
			\item Il sito dev'essere accessibile da parte di categorie d'utenti diversificate ed utilizzando dispositivi di vario tipo, compresi smartphones e tablet;
			\item Il sito deve presentare possibilit� di stampa flessibile a seconda della pagina richiesta;
			\item Il sito dev'essere visitabile tramite i seguenti browser: 
			\begin{itemize}
				\item Firefox 3.6;
				\item Internet Explorer dalla versione 7 alla versione 11; Edge 13;
				\item Chrome 14;
				\item Opera 12.16;
				\item Safari 9.
			\end{itemize}
			\item Separazione tra struttura, presentazione, comportamento;
			\item Conformit� agli standard W3C per XHTML, CSS, JS;
			\item Sito comprensibile da screen-reader.
		\end{itemize}
	}
	\subsection{Requisiti}{
		Di seguito sono presentati i requisiti emersi dall'analisi iniziale e quelli che si sono aggiunti nel corso dello svolgimento del progetto. Ciascuno � identificato da un numero progressivo per semplificarne l'individuazione successiva.\\
		\newcounter{magicrownumbers}
		\newcommand\rownumber{\stepcounter{magicrownumbers}\arabic{magicrownumbers}}
		\begin{table}[h]
			\centering
			\begin{tabular}{|p{\dimexpr 0.15\linewidth-2\tabcolsep}|p{\dimexpr 0.8\linewidth-2\tabcolsep}|}
				\hline
				\textbf{ID Req.} & \textbf{Descrizione}\\
				\hline
				\centering \rownumber	&	Il sito dev'essere visualizzabile sui browser elencati all'interno di "Vincoli generali"\\
				\hline
				\centering \rownumber	&	Il sito dev'essere accessibile indipendentemente dalla grandezza dello schermo del dispositivo\\
				\hline
				\centering \rownumber	&	Il sito dev'essere fruibile anche senza richiedere un foglio di stile\\
				\hline
				\centering \rownumber	&	Le figure significative dovranno essere comprensive di un attributo alt per favorire l'accesso ad utenti non vedenti\\
				\hline
				\centering \rownumber	&	Ai tag quali <input> e <textarea> devono essere associati tabindex e accesskey\\
				\hline
				\centering \rownumber	&	Le gradazioni di colori non devono risultare sgradevoli o di intralcio a persone affette da daltonismo\\
				\hline
				\centering \rownumber	&	Il layout deve risultare fluido nel ridimensionamento del carattere tramite i tasti Ctrl + e Ctrl -\\
				\hline
				\centering \rownumber	&	Il sito deve essere validato per la parte di XHTML2.0, CSS3 e secondo gli standard WAI\\
				\hline
			\end{tabular}
			\label{tab:requisiti}
			\caption{Elenco dei requisiti}
		\end{table}
	}
}

\newpage



\end{document}
