\section{Suddivisione dei lavori}{
	Per realizzare il progetto abbiamo cercato di distribuire il carico di lavoro in modo quanto più possibile uniforme, dividendo il progetto in sezioni indipendenti così da procedere nello sviluppo in modo quanto più possibile parallelo e libero da conflitti.
	
	La suddivisione è stata la seguente: 	
	\begin{itemize}\itemsep1pt
		\item \textbf{Sebastiano Marchesini} 
		\begin{itemize}\itemsep1pt
			\item Title, Meta-Title e Keywords;
			\item Creazione fogli di stile (home.css, print.css, mobile.css, expolerer.css);
			\item Creazione logo, footer del sito;
<<<<<<< HEAD
			\item Creazione pagine \texttt{Realizzazioni}, \texttt{Contattaci}, \texttt{Chi Siamo}, \texttt{Home};
			\item Accessibilità;
			\item Stesura sezione Introduzione, Associazioni, Classi, Linguaggi Utilizzati, Architettura, Presentazione nella relazione.
=======
			\item Accessibilità;
			\item Stesura sezione Introduzione, Architettura, Classi, Associazioni, Progettazione logica, Presentazione nella relazione.
>>>>>>> origin/master
		\end{itemize}
		\item \textbf{Andrea Grendene}
		\begin{itemize}\itemsep1pt
			\item Creazione e modifica del database MySQL;
			\item Gestione sessioni;
			\item Comportamento del sito tramite PHP;
			\item Stesura sezioni Linguaggi utilizzati, Struttura, Gestione dati, PHP nella relazione.
		\end{itemize}
		\item \textbf{Pietro Gabelli}
		\begin{itemize}\itemsep1pt
			\item Struttura del sito tramite PHP;
			\item Gestione errori;
			\item Comportamento del sito (JS);
			\item Accessibilità;
			\item Testing;
			\item Stesura delle sezioni Descrizione Generale, Suddivisione lavori, Accessibilità, Verifica e test nella relazione.
		\end{itemize}
	\end{itemize}
}