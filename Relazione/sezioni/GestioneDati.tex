\section{Gestione Dati}{
	\subsection{Introduzione}
		Il database è stato scritto con SQL, un tipo di linguaggio che permette di modificare e interrogare la base di dati con facilità e sicurezza. Tale tecnologia infatti permette di inserire, modificare, cancellare ed eseguire altre operazioni attraverso dei comandi specifici, eseguiti tramite delle istruzioni apposite di PHP. I dati salvati riguardano vari aspetti del sito, di cui la maggior parte viene usata per fornire dei servizi aggiuntivi all'amministratore.
	
	\subsection{Descrizione delle entità}{
		Ogni entità ha un proprio codice identificativo, anche quando si potevano usare altri attributi come chiave primaria. In questo modo il sistema è solido anche in previsioni di modifiche future, ad esempio perché cambiano le necessità dell'azienda o si presentano dei casi particolari. Di seguito verranno descritte brevemente le entità utilizzate.
			\begin{itemize}\itemsep1pt
				\item \textbf{Clienti}: rappresenta il cliente del Salone, i dati che contiene sono:
					\begin{itemize}\itemsep1pt
						\item CodCliente: INT, chiave primaria;
						\item Nome: VARCHAR(30);
						\item Cognome: VARCHAR(30);
						\item Telefono: VARCHAR(10);
						\item Email: VARCHAR(50);
						\item DataNascita: DATE, il formato salvato in questo caso è del tipo yyyy-m-gg, dove `y' rappresenta le cifre degli anni, `m' quelle dei mesi e `g' quelle dei giorni.
					\end{itemize}
				\item \textbf{TipoAppuntamento}: rappresenta il tipo di lavoro che il cliente può chiedere al Salone di fare. I dati contenuti sono:
					\begin{itemize}\itemsep1pt
						\item CodTipoAppuntamento: SMALLINT, chiave primaria;
						\item NomeTipo: VARCHAR(30), ovvero il nome dell'intervento;
						\item Costo: DOUBLE, rappresenta il prezzo del lavoro;
						\item Sconto: DOUBLE, è lo sconto che il responsabile può decidere di applicare al tipo di intervento, a seguito per esempio di una promozione o di un anniversario. 
					\end{itemize}
				\item \textbf{Appuntamenti}: rappresenta la prenotazione del cliente per fissare un orario in cui effettuare un determinato intervento. I dati che contiene sono:
					\begin{itemize}\itemsep1pt
						\item CodAppuntamento: INT, chiave primaria;
						\item CodCliente: INT, chiave esterna;
						\item DataOra: DATETIME, è un formato che contiene una data e un'ora, la prima è strutturata in modo uguale a DataNascita di Clienti, l'ora invece ha un formato del tipo hh:mm:ss, dove `h' rappresenta le cifre dell'ora, `m' quelle dei minuti e `s' quelle dei secondi;
						\item CodTipoAppuntamento: INT, chiave esterna.
					\end{itemize}
				\item \textbf{Prodotti}: rappresenta il prodotto che il Salone usa e vende durante gli appuntamenti. I dati contenuti sono:
					\begin{itemize}\itemsep1pt
						\item CodProdotto: INT, chiave primaria;
						\item Nome: VARCHAR(20), è il nome che identifica il prodotto;
						\item Marca: VARCHAR(30), rappresenta il nome dell'azienda che crea il prodotto;
						\item Tipo: VARCHAR(30), è il tipo di utilizzo a cui il prodotto è destinato;
						\item Quantita: INT, è un valore che distingue i prodotti uguali ma con contenitori di dimensioni diverse, ad esempio per uno stesso tipo di shampoo possono esserci una confezione da 100 ml e un'altra da 250 ml;
						\item Prezzo: DOUBLE, indica quanto il Salone spende per comprare il singolo prodotto;
						\item PRivendita: DOUBLE, indica quanto il cliente deve spendere per acquistare il prodotto dal Salone.
					\end{itemize}
				\item \textbf{ProdApp}: collega i prodotti agli appuntamenti in cui essi vengono usati, i dati che contiene sono:
					\begin{itemize}\itemsep1pt
						\item CodAppuntamento: INT, chiave primaria e esterna;
						\item CodProdotto: INT, chiave primaria e esterna;
						\item Utilizzo: INT, rappresenta quanto prodotto viene usato durante l'appuntamento, come ad esempio 30 ml di shampoo.
					\end{itemize}
				\item \textbf{Account}: rappresenta l'account che un amministratore possiede e con cui può accedere al sito. I dati che contiene sono:
					\begin{itemize}\itemsep1pt
						\item CodAccount: SMALLINT, chiave primaria;
						\item username: VARCHAR(20);
						\item password: VARCHAR(32), contiene la versione cifrata della password dell'utente.
					\end{itemize}
				\item \textbf{Messaggi}: rappresenta il messaggio che il cliente può scrivere all'azienda, tramite l'apposito form presente nel sito. I dati contenuti sono:
					\begin{itemize}\itemsep1pt
						\item CodMessaggi: INT, chiave primaria;
						\item CodCliente: INT, chiave esterna;
						\item Contenuto: VARCHAR(512);
						\item DataOra: DATETIME, rappresenta la data e l'ora in cui il server riceve il messaggio;
						\item ToRead: BOOLEAN, è un flag che indica se il messaggio è già stato letto o no.
					\end{itemize}
				\item \textbf{Images}: contiene i dati relativi alle immagini presenti nella galleria fotografica di \href{http://tecnologie-web.studenti.math.unipd.it/tecweb/~pgabelli/public\_html/foto.php}{public\_html/foto.php}. I dati che contiene sono:
					\begin{itemize}\itemsep1pt
						\item Img\_title: INT, chiave primaria;
						\item Img\_desc: VARCHAR(200), rappresenta la descrizione associata all'immagine;
						\item Img\_filename: VARCHAR(160), è il nome dell'immagine, serve per poter eseguire tutte le operazioni di eliminazione e modifica dell'immagine.
					\end{itemize}
				\item \textbf{Contatori}: è una View, ossia è una specie di entità i cui dati non vengono inseriti direttamente dall'utente, bensì sono ricavati dai valori di altre entità. In questo caso le informazioni riportate riguardano la statistica degli appuntamenti in base al tipo, gli attributi previsti sono:
					\begin{itemize}
						\item Parziali BIGINT, è il numero di appuntamenti del tipo con CodTipoAppuntamento uguale all'attributo Tipo;
						\item Tipo SMALLINT, è il codice del tipo di appuntamento di cui si vuole visualizzare la statistica.
					\end{itemize}
			\end{itemize}
	}
	\subsection{Rapporto tra le entità} {
		Le entità ovviamente sono legate in maniera diversa tra loro:
		\begin{itemize}
			\item Clienti, TipoAppuntamento, Prodotti, Account e Images non contengono chiavi esterne, i primi tre sono collegati ad altre entità descritte nei punti successivi, gli altri due invece non possiedono alcun legame con le altre entità.
			\item Appuntamenti contiene come chiavi esterne il codice di Clienti, perché un appuntamento è legato ad un unico cliente e ogni cliente può avere più appuntamenti, e il codice di TipoAppuntamento, perché un appuntamento è di un solo tipo ma possono esserci più appuntamenti dello stesso tipo.
			\item ProdApp contiene come chiavi esterne il codice di Prodotti e quello di Appuntamenti, in quanto rappresenta un'entità intermedia che le collega tra loro, dato che possono esserci più prodotti associati ad un appuntamento e lo stesso prodotto può essere usato in più appuntamenti.
			\item Messaggi contiene come chiave esterna il codice di Clienti, perché ogni messaggio è associato ad un unico cliente e ogni cliente può scrivere più messaggi.
		\end{itemize}
	}
	\subsection{Codice PHP per l'interazione con il database}{
	\label{sec:PHPDB}
		Quasi tutte le funzioni PHP che gestiscono l'interazione tra il sito e il database sono contenute nel file ‘‘DBlibrary.php’’ dentro la cartella ‘‘utils’’. Tra di esse le più importanti sono ‘‘dbconnect’’, in cui sono contenute tutte le istruzioni per poter collegarsi al database SQL, in modo da poter eseguire facilmente le query, e ‘‘cleanString’’, che esegue delle pulizie di base sulla stringa passata come parametro, ovvero trasforma i caratteri speciali in entità HTML ed elimina gli spazi e le tabulazioni che possono essere presenti all'inizio o alla fine della stringa.\\
		Tutte le altre funzioni invece rappresentano delle query da dover eseguire sul database, a parte un paio di esse che sono di supporto. È possibile suddividerle in alcuni gruppi, dato che molte di esse lavorano allo stesso modo ma su entità diverse: le funzioni del tipo ‘‘lista’’ (listaClienti, listaAppuntamenti, eccetera) prelevano dal database tutti i dati relativi all'entità a cui fanno riferimento, li salvano in un array di oggetti e restituiscono quest'ultimo all'utente; le funzioni del tipo ‘‘aggiungi’’ inseriscono una nuova istanza nel database in base ai parametri forniti alla funzione, ovviamente non prima che essi vengano controllati; le funzioni del tipo ‘‘elimina’’ cancellano un'istanza del database in base al codice fornito; le funzioni del tipo ‘‘aggiorna’’ modificano un'istanza in base al codice e ai parametri forniti, dopo che questi ultimi vengono controllati; le funzioni ‘‘mostra’’ restituiscono i dati relativi ad una precisa istanza, in base al codice fornito alla funzione; infine le altre funzioni eseguono delle query più specifiche, che non hanno nulla in comune né tra loro né con le funzioni descritte prima.\\
		I tipi restituiti si possono dividere in due categorie: la prima è rappresentata da un singolo oggetto o un array di oggetti, dove il valore che segnala la presenza di errori a chi chiama la funzione è ‘‘NULL’’; la seconda è rappresentata da un valore booleano, dove ‘‘TRUE’’ indica che l'esecuzione della query è andata a buon fine, mentre ‘‘FALSE’’ segnala che ci sono stati dei problemi. Questi ultimi possono essere di vario tipo, ad esempio se il database non è accessibile, se la connessione si interrompe all'improvviso o se la query eseguita non è corretta.
		I controlli sui dati vengono effettuati in ogni funzione che prevede dei parametri, perché ciascuno di essi generalmente viene ottenuto da un input richiesto all'utente, che può essere errato e quindi può generare problemi. In particolare se JavaScript è disattivato avere questi controlli è fondamentale, in quanto non c'è alcuna garanzia che il formato dei dati non faccia danni. Un altro problema è causato dai caratteri speciali, che non possono essere salvati nel database e quindi devono essere trasformati nella corrispondente entità HTML; questa operazione inoltra aiuta la stampa degli oggetti nelle pagine HTML, perché i caratteri speciali risultano essere già in una forma corretta. Nelle stringhe inoltre viene controllato se ci sono degli spazi all'inizio o alla fine, perché sono caratteri inutili che possono comparire per vari motivi, come un errore dell'utente, e possono provocare dei problemi, ad esempio nel database le dimensioni degli attributi sono limitate e quindi dei caratteri di troppo possono rischiare di tagliare una parte dei dati inseriti dall'utente. Per i prezzi viene fatto un controllo di formato apposito, in quanto possono avere 0, 1, 2 cifre dopo la virgola, ma non di più, mentre la parte prima può essere lunga quanto serve; inoltre può essere presente la virgola al posto del punto, in tal caso essa viene sostituita con il punto, garantendo la riuscita della query.\\
		Solo i parametri di tipo stringa vengono controllati, dato che in genere gli altri tipi segnalano errore se sono presenti dei caratteri errati; ad esempio se l'utente deve inserire un numero e aggiunge un carattere speciale allora la query fallirà, in quanto viene inviato un valore di tipo stringa dove è atteso un numero. Un altro caso particolare riguarda le date, che subiscono un controllo del formato molto più severo e quindi non vengono verificate con i metodi delle altre stringhe, in quanto i casi in cui tali istruzioni sono necessarie non passano i controlli usati dalle date. Infine gli attributi ‘‘Telefono’’ di ‘‘Cliente’’ e ‘‘Img\textunderscore filename’’ di ‘‘Images’’ subiscono un'ulteriore operazione: vengono individuati ed eliminati tutti gli spazi contenuto al loro interno; questo perché altrimenti il loro formato risulterebbe errato rispetto a quello previsto.
	}
}