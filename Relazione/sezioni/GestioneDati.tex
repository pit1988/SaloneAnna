\section{Gestione Dati}{
	\subsection{Introduzione}{
		Il database è stato scritto con SQL, un tipo di linguaggio che permette di modificare e interrogare la base di dati con facilità e sicurezza. Tale tecnologia infatti permette di inserire, modificare, cancellare ed eseguire altre operazioni attraverso dei comandi specifici, eseguiti tramite delle istruzioni apposite di PHP. I dati salvati riguardano vari aspetti del sito, di cui la maggior parte viene usata per fornire dei servizi aggiuntivi all'amministratore.
	}
	\subsection{Descrizione delle entità}{
		Ogni entità ha un proprio codice identificativo, anche quando si potevano usare altri campi come chiave primaria. In questo modo il sistema è solido anche in previsioni di modifiche future, ad esempio perché cambiano le necessità dell'azienda o si presentano dei casi particolari. Di seguito verranno descritte brevemente le entità utilizzate.
		\begin{itemize}
		\item Clienti contiene i dati dei clienti, ovvero il nome, il cognome, il telefono, l'email e la data di nascita; i primi quattro campi sono delle stringhe, mentre la data è di tipo Date, un tipo che salva il valore nel formato yyyy-mm-gg, dove 'y' rappresenta le cifre degli anni, 'm' quella dei mesi e g quella dei giorni.
		\item TipoAppuntamento contiene i possibili tipi di appuntamento, quindi possiede come campi il nome, di tipo stringa, il costo del servizio e lo sconto che l'amministratore vuole applicare, entrambi di tipo numerico.
		\item Appuntamenti contiene i dati relativi ai singoli appuntamenti, quindi il codice del tipo, il codice del cliente e l'orario prenotato; quest'ultimo è di tipo DateTime, ovvero contiene la data, nel formato descritto in Clienti, e l'ora, rappresentata nel formato hh:mm:ss, dove 'h' rappresenta le cifre dell'ora, 'm' quelle dei minuti e 's' quelle dei secondi.
		\item Prodotti contiene i dati di ogni prodotto presente nel magazzino dell'azienda, è caratterizzato dai campi nome, marca, tipo, quantità, prezzo e prezzo di rivendita. ‘‘Nome’’ indica il nome che identifica l'oggetto, ‘‘marca’’ è il nome dell'azienda che vende il prodotto, ‘‘tipo’’ indica per quale uso è destinato l'oggetto descritto, ‘‘quantità’’ distingue i prodotti che differiscono fra loro solo per la dimensione del contenitore, ad esempio quando ho un flacone che contiene 250 ml di prodotto e un'altro che ne contiene 400 ml, ‘‘prezzo’’ indica quando costa l'acquisto del prodotto, ‘‘prezzo di rivendita’’ infine rappresenta il prezzo che deve pagare il cliente per acquistare quel prodotto. Nome, marca e tipo sono stringhe, invece quantità, prezzo e prezzo di rivendita sono di tipo numerico. 
		\item ProdApp contiene i prodotti usati durante gli appuntamenti, permette così di memorizzare quali e quanti prodotti sono stati usati per ogni appuntamento; i dati che lo caratterizzano sono il codice dell'appuntamento, il codice del prodotto, che insieme rappresentano la chiave primaria dell'entità, e la quantità utilizzata, ovvero quanto prodotto è stato usato durante l'appuntamento. Tutti i campi sono di tipo numerico.
		\item Account contiene i dati relativi agli account degli amministratori, viene usato quando l'utente tenta di eseguire la login nel sito. I campi che contiene sono username e password, entrambi di tipo stringa.
		\item Messaggi contiene i messaggi inviati agli amministratori. Essi sono caratterizzati dal codice del cliente, di tipo intero, dal contenuto, di tipo stringa, l'orario in cui vengono ricevuti dal server, di tipo DateTime, e dal valore booleano ToRead, che indica se il messaggio è ancora da leggere.
		\item Immagini contiene le informazioni relative alle immagini caricate nella pagina ‘‘Foto’’ del sito, i campi di cui è composta sono la descrizione e il nome, entrambi di tipo stringa. In particolare la descrizione è il messaggio che viene stampato sulla pagina per indicare all'utente cosa è rappresentato nella foto, mentre il nome serve per le operazioni di modifica o eliminazione del file.
		\item Contatori è una View, ossia è una specie di entità i cui dati non vengono inseriti direttamente dall'utente, bensì vengono ricavati dai valori di altre entità. In questo caso le informazioni riportate riguardano la statistica degli appuntamenti in base al tipo: i campi previsti sono quindi ‘‘Tipo’’, dove viene salvato il codice dei tipi di appuntamento, e ‘‘Parziali’’, dove viene salvato il numero di appuntamenti di quel tipo presenti nel database.
		\end{itemize}
	}
}