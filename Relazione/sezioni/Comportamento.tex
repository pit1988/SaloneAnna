\section{Comportamento}{
	Le funzionalità offerte tramite Javascript sono: 
	\begin{itemize}\itemsep1pt
		\item Il pulsante per risalire ad inzio pagina, creato tramite una funzione jQuery;
		\item L'inserimento della mappa nella pagina "contattaci" ottentuta aggiungendo all'immagine nella pagina, la cartina fornita da Google.
		\item Il controllo lato client dei dati inseriti nella form nelle pagine Contattaci, nelle pagine da cui modificare e creare un appuntamento od un cliente, nella pagina da cui creare un prodotto ed in quelle da cui far iniziare una ricerca su di un cliente od un giorno particolare. Questo è stato fatto creando delle matrici per ogni form che si desiderava controllare, al cui interno erano riportati i nomi dei dati e le espressioni regolari che ne stabilivano la validità; fatto questo, sono state abbinate a delle funzioni che data la matrice dei dati, si occupavano d'invocare una funzione generica che s'occupa di reperire il dato all'interno della pagina, effettuare il controllo e segnalare gli errori trovati. Le funzioni invocate dalle form si occupano di impedire l'invio dei dati segnalando anche la presenza di errori vicino al pulsante d'invio della form.
	\end{itemize}
	Sapendo che Javascript più può non essere disponibile od essere stato disabilitato dall'utente:
	\begin{itemize}\itemsep1pt
		\item abbiamo sviluppato le pagine in modo da fornire messaggi d'errore qualora le invocazioni delle funzioni PHP che s'occupano della modifica dei dati non vadano a buon fine; questo per tutte le pagine che prevedevano un'interazione con l'utente.
		\item inserito all'interno della pagina contattaci, un tag \textit{<noscript>} contenente degli aiuti alla compilazione della form.
	\end{itemize}
	Avendo utilizzato JS per un limitato numero di funzionalità la cui assenza non preclude l'uso del sito (la mappa è presente come immagine, mentre gli errori sugli inserimenti sono segnalati in ogni caso), riteniamo di aver raggiunto l'obiettivo di garantire un degrado elegante delle pagine, in assenza del supporto a Javascript da parte del browser in uso.
}