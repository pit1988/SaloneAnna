\section{Accessibilità}{
	\subsection{Implementazione}{
	Al fine di garantire l'utilizzo del sito ad utenti con disabilità, si sono:
	\begin{itemize}\itemsep0.5pt
		\item Validati i file che compongono il sito con i validatori XHTML1.0-Strict e CSS3 del W3C;
		\item Separata struttura, presentazione e comportamento;
		\item Fonita la possibilità per la tecnologia assistiva di ignorare contenuti non testuali puramente decorativi;
		\item Potuti ridimensionare i testi fino al 200\%, senza perdita di contenuto e funzionalità;
		\item Creati contenuti rappresentabili in modalità differenti senza perdita d'informazioni o struttura;
		\item Utilizzato un rapporto di contrasto di almeno: 
		\begin{itemize}\itemsep1pt
			\item 7:1 tra testo e sfondo;
			\item 4,5:1 tra testo grande ed immagini contenenti testo grande nella home;
		\end{itemize}
		\item Reso le funzionalità del sito utilizzabili tramite tastiera, mediante i tabindexs e accesskeys;
		\item Inseriti:
		\begin{itemize}\itemsep0.5pt
			\item informazioni relative alla posizione dell'utente (breadcrumbs); 
			\item titoli appropriati  per le pagine web; 
			\item testo appropriato e testo alternativo per i collegamenti;
			\item per quanto riguarda le tabelle, è stato:
				\begin{itemize} \itemsep0.5pt
					\item Aggiunto attributo \textbf{summary} ed elemento \textbf{caption};
					\item Aggiunti THEAD, TBODY; per tabelle che si prevedeva particolarmente lunghe anche TFOOT.
					\item Inseriti elementi \textit{th} per le intestazioni e riservati \textit{td} per i soli dati.
					\item Utilizzato l'attributo \textbf{scope} per rendere esplicita la relazione tra le celle d'intestazione e le celle di dati: indicando l’estensione di celle per cui l’intestazione si trova nella cella marcata con l’attributo \textbf{scope}. \\
					Abbiamo marcanto \textbf{scope} con il valore \textit{col} le celle d'intestazione, assicurando che le informazioni di intestazione si applichino a tutte le celle della colonna; col valore \textit{colspan}, invece, le celle d'intestazione relative ad un gruppo di colonne.
					\item Utilizzato serie di \textbf{id} e di \textbf{headers}: il primo per identificare una cella che contiene informazioni di intestazione a cui fare riferimento con headers serve per risalire alle celle d’intestazione da porre in relazione con la cella corrente. \\
					Questo per aumentare la comprensione dei contenuti, diminuire il disorientamento ed agevolare l'utente che fa uso di screen reader, visto che questi ultimi che tengono traccia delle intestazioni.
				\end{itemize}
			\item alternative testuali per il contenuto non testuale:
				\begin{itemize} \itemsep0.5pt
					\item corredata ogni immagine portatrice di contenuto, con l'attributo alt che la descrive;
					\item aggiunta di una label ad ogni campo di input della form, in aiuto dello screen reader; corredata ogni form di una legend, per i medesimi motivi.
				\end{itemize}
			\item intestazioni ed etichette appropriate; 
			\item l'indicatore del focus nelle interfacce utilizzabili da tastiera; 
			\item lingua predefinita per il contenuto delle pagine; usati gli attributi \texttt{xml:lang} per definire parole o blocchi in lingua diversa da quella predefinita della pagina; non sono state utilizzate abbreviazioni od acronimi;
 			\item definiti i tag meta: \textit{Description, Keywords, Copyright, Author}; le parole chiave contenute nel tag "keywords" sono in lingua italiana, visto il pubblico a cui il sito si rivolge.
 			\item È stato definito infine un link di ritorno ad inizio pagina.
		\end{itemize} 
		\item Mantenuto un meccanismo di navigazione coerente all'interno delle pagine web del sito, grazie al template ottenuto con le funzioni PHP;
		\item Individuati eventuali errori di inserimento e descritti con del testo; fornite etichette (o istruzioni) per l'input dell'utente.
		\item Inserite modalità per saltare i blocchi di contenuto che si ripetono su più pagine;
	\end{itemize}
	Inoltre, non si sono:
	\begin{itemize}\itemsep1pt
		\item inseriti contenuti audio e video (contenuti multimediali basati sul tempo);
		\item posti vincoli di tempo all'utente per consultare i contenuti o compilare i campi dati;
		\item utilizzato il colore come modalità visiva per rappresentare le informazioni, indicare azioni, elemento di distinzione visiva; inserito contenuto audio eseguito automaticamente all'interno della pagina;
		\item sviluppati contenuti che possano causare attacchi epilettici (non s'è inserito contenuto lampeggiante);
		\item inseriti cambiamenti del contesto su alcun componente che riceve il focus;
	\end{itemize}
		Al fine di facilitare l'utilizzo del sito da parte di utenti con disabilità, si è:
		\begin{itemize}\itemsep1pt
			\item aiutata la navigazione tra le pagine, con le seguenti tecniche:
			\begin{itemize}\itemsep1pt
				\item \textit{un path o breadcrumb}, per individuare il contesto;
				\item \textit{un menù di link} per mostrare dove si può andare;
				\item \textit{uso dello stesso stile} per tutti i link del sito;
				\item \textit{link di ritorno ad inizio pagina}.
				\item ridefiniti i tabindex per la navigazione tra i vari link.
				\item definite scorciatoie con \textit{accesskey} per la navigazione tra le pagine.
			\end{itemize}
			\item Mantenuti chiari i link.
			\item Creato scorciatoie per dispositivi mobili per favorire la navigazione.
		\end{itemize}
	}
	\subsection{Combinazione dei colori}{
		È stata utlizzato uno schema di colori che garantisse un contrasto di almento 7:1 tra sfondo e testo; per testare le scelte fatte è stato utilizzato il servizio offerto da \url{http://wave.webaim.org}.\\
		Il servizio offerto da \url{http://colorfilter.wickline.org} ha permesso di capire come utenti con disturbi visivi visualizzino il sito.\\
		Di seguito vengono riportati i risultati ottenuti sulla home page.
%		\newpage
		\begin{figure}[H]
%			\centering
			\begin{subfigure}[b]{0.5\textwidth}
				\includegraphics[width=\textwidth]{\docsImg Home.png}
				%\vspace{-40pt}
				\caption{Home Page originale}
				\label{Home Page originale}
			\end{subfigure}
			\begin{subfigure}[b]{0.5\textwidth}
				\includegraphics[width=\textwidth]{\docsImg Deuteran.png}
				%\vspace{-40pt}
				\caption{Home Page vista da un deutranope}
				\label{Home Page vista da un deutranope}
			\end{subfigure}
			\\
			\\
			\begin{subfigure}[b]{0.5\textwidth}
				\includegraphics[width=\textwidth]{\docsImg Protean.png}
				%\vspace{-40pt}
				\caption{Home Page vista da un protranope}
				\label{Home Page vista da un protranope}
			\end{subfigure}
			\begin{subfigure}[b]{0.5\textwidth}
			\includegraphics[width=\textwidth]{\docsImg Tritan.png}
				%\vspace{-40pt}
				\caption{Home Page vista da un tritranope}
				\label{Home Page vista da un tritranope}
			\end{subfigure}
			\caption{Homepage vista da persone con problemi nel distinguere i colori}
			\label{fig: Homepage vista da persone con problemi nel distinguere i colori}
		\end{figure}
	}
}
