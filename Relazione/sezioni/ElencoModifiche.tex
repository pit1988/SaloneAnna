\subsection{Elenco Modifiche}{
	Di seguito riportiamo l'elenco delle modifiche effettuate dopo la correzione seguita alla consegna di settembre 2016:
	\begin{itemize}\itemsep1pt
		\item HTML:
		\begin{itemize}\itemsep1pt
			\item validato il codice di tutte le pagine;
			\item inseriti tag title, description appropriati;
			\item inseriti link di supporto alla navigazione per il salto degli elementi;
			\item inserite sezioni <noscript>;
			\item Ricontrollo completo meta tag nell’head.
			\item Ricontrollo dei tag html nel body.
			\item Particolare attenzione tag dedicati allo screen reader e indirizzati a un web semantico.
			\item Inseriti tab index e accesskey.
			\item Inserimento elementi e link per lo screen reader
		\end{itemize}
		\item CSS:
		\begin{itemize}\itemsep1pt
			\item minificato il codice CSS
			\item validato con i validatori W3C;
			\item Rivisitazione completa del layout in ogni pagina.
			\item Particolare attenzione grandezza degli elementi a seconda della risoluzione dello schermo.
			\item Testo e elementi portati per la maggior parte ad unità di misura flessibili come “em” e “/%”
			\item Cambiati colori principali dei pulsanti unificandoli.
			\item tag specifici per ogni elemento (es. address).
			\item unificazione dei fogli di stile in home.css .
			\item correzione foglio di stile per la stampa.
			\item creazione del foglio di stile per dispositivi mobile ( e conseguente aggiunta elementi html per favorire la navigazione mobile ).
			\item creazione del foglio di stile dedicato al browser Internet Explorer.
		\end{itemize}
		\item Javascript:
		\begin{itemize}\itemsep1pt
			\item validato il codice;
			\item minificato il file;
			\item aggiunta funzione per il controllo degli input dinamici prezzo-formato, dato-formato;
			\item aggiunte funzioni di supporto al caricamento controlli e dati per le maschere di modifica generate tramite cgi.
		\end{itemize}
		\item XML
		\begin{itemize}\itemsep1pt
			\item aggiunti i file ‘‘profili.xml’’, ‘‘profili.xsd’’ e ‘‘search.xslt’’;
			\item modificati i file xslt in modo da seguire il layout e le regole segnalate nella sezione precedente ‘‘HTML’’;
			\item modificati gli schemi xsd secondo il modello Tende alla veneziana.
		\end{itemize}
		\item PERL
		\begin{itemize}\itemsep1pt
			\item reso il codice prodotto valido per XHTML1.0 Strict;
			\item aggiunto il codice per l'invio automatico dell'email;
			\item aggiunti i controlli sui dati inviati per l'invio dell'email, per l'accesso e la modifica delle credenziali dell'amministratore, per l'inserimento e la modifica dei prodotti venduti;
			\item controllato, aggiustato e verificato tutto il codice prodotto.
		\end{itemize}
		\item Relazione: 
		\begin{itemize}\itemsep1pt
			\item corrette credenziali d'accesso;
			\item modificate tutte le sezioni in seguito a modifiche sul codice.
		\end{itemize}
		\item Testing:
		\begin{itemize}\itemsep1pt
			\item Provato il sito sui browser riportati;
			\item Effettuati test accessibilità e corretti errori trovati (contrasto, label, legends, tabindex ed accesskey);
			\item Verificata mobile-friendliness del sito;
			\item controllo e correzione in caso di ingrandimento della pagina fino al 200%.
		\end{itemize}
	\end{itemize}
}