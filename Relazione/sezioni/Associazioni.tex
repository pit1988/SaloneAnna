\section{Associazioni}{
	Le Associazioni (o Relazioni) rappresentano legami logici fra due o più entità dell'applicazione.
	
	Le Associazioni create al fine di modellare il progetto sono: 	
	\begin{itemize}\itemsep1pt
		\item \textbf{Appuntamento - Clienti} 
		\begin{trivlist}\itemsep1pt
			\item Ogni cliente ha uno o più appuntamenti (o nessuno);
			\item Ogni Appuntamento ha un cliente;
			\item Molteplicità 1:N;
			\item Totalità:
			\begin{itemize}
				\item Parziale verso Appuntamenti;
				\item Totale verso Clienti;
			\end{itemize}
		\end{trivlist}
		\item \textbf{Appuntamento - Prodotti} 
		\begin{trivlist}\itemsep1pt
			\item Durante un Appuntamento possono venir usati più prodotti (o nessuno);
			\item Ogni prodotto può essere usato in più appuntamenti in diverse quantità (per cliente, o anche lo stesso cliente in appuntamenti diversi);
			\item Molteplicità N:M;
			\item Totalità:
			\begin{itemize}
				\item Parziale verso Prodotti;
				\item Parziale verso Appuntamenti.
			\end{itemize}
		\end{trivlist}
		\item \textbf{TipoAppuntamento - Appuntamento}
		\begin{trivlist}\itemsep1pt
			\item Ogni TipoAppuntamento può essere usato su più appuntamenti;
			\item Un Appuntamento ha un e uno solo TipoAppuntamento;
			\item Molteplicità 1:N;
			\item Totalità:
			\begin{itemize}
				\item Parziale verso Appuntamento;
				\item Totale verso TipoAppuntamento.
			\end{itemize}
		\end{trivlist}
	\end{itemize}
}