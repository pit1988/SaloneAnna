\section{Architettura}{
	Il layout è stato strutturato allo scopo di rendere il sito fruibile indipendentemente dal dispositivo, definendo:
	\begin{itemize}\itemsep1pt
		\item layout per dispositivi desktop;
		\item layout per dispositivi mobili;
		\item layout di stampa.
	\end{itemize} 
	
	\subsection{Progettazione layout}{
		Si è deciso di utilizzare un layout di tipo responsive a singola colonna principale, adattabile secondo la dimensione dello schermo, impostando un limite sulla larghezza per passare al layout per dispositivi mobili.
	}
	\subsection{Sviluppo layout}{
		Nell'immagine che segue viene mostrata la struttura data ai vari blocchi \textit{div} che compongono il sito, in cui sono contenute le informazioni divise per area tematica le informazioni.
	\\
	Il layout fluido orizzontalmente si adatta in base alla larghezza dello schermo, con l'obiettivo di far sviluppare verso il basso il sito.
		\begin{figure}[H]
			\includegraphics{\docsImg Schemasito.png}
			\caption{Schema del sito}
			\label{Schema del sito}
		\end{figure}
		
		Il foglio di stile standard viene utilizzato sui browser per computer desktop e portatili, fino a che la larghezza dello schermo rimane maggiore di 650 px: al di sotto si passa automaticamante ad usare il CSS per al mobile.
		\\
		Analizzando il sito spostandosi dall'alto verso il basso, gli elementi che si incontrano sono:
		\begin{itemize}\itemsep1pt
			\item Il div \textbf{header}, informa l'utente su ciò che sta visitando; comprende:
			\begin{itemize}\itemsep1pt
				\item Titolo e logo; nella pagina relativa alle vendite è presente inoltre il pulsante d'accesso all'area riservata del sito, che conduce alla pagina da cui effetuare il login.
				\item Il div \textbf{breadcrumbs}: aiuta l'utente ad identificare la posizione in cui si trova all'interno del sito, a partire dalla homepage;
				\item Il menù utente, identificato con il div \textbf{menu}, presenta scelte differenziate a seconda che si abbia effettuato l'accesso. Agli utenti non autenticati sono disponibili i collegamenti a \textit{Home, Foto, Chi Siamo, Prezzi, Contattaci}. 
				Dopo aver effettuato l'accesso si hanno a disposizione le seguenti scelte \textit{Home, Prezzi, Immagini, Prodotti, Clienti, Appuntamenti, Utilità}.
				Viene evidenziata la posizione corrente e si trova in posizione centrale rispetto alla pagina; si estende orizzontalmente.
				Per agevolare lo sviluppo questa parte viene costruita da funzioni PHP;
			\end{itemize}
			\item Segue il contenuto vero e proprio della pagina, inserito nel div \textbf{content}: ha il compito di esporre le informazioni che si stanno trattando;
			\item Alla fine della pagina, il \textbf{footer}, costruito con un layout a 2 colonne; al suo interno sono presenti:
			\begin{itemize}\itemsep1pt
				\item i principali riferimenti all'azienda;
				\item un piccolo logo dell'azienda ed i simboli di validità per l'XHTML ed il CSS;
			\end{itemize}
			Anche il footer è stato costruito con una funzione, in modo da ottenere un template valido per tutte le pagine del sito e potersi concentrare nello sviluppo del contenuto delle pagine.
		\end{itemize}
		}
	\subsection{Layout per dispositivi mobili}{
		Il layout per dispositivi mobili è stato sviluppato in modo da favorire l'incolonnamento degli elementi, rimuovendo quanto possibili margini e padding, sfruttando al meglio l'area disponibile; sono state infine ridimensionate le immagini presenti.
	}
	\subsection{Layout di stampa}{
		Nel layout di stampa sono stati tolti gli elementi che non portavano informazioni significative; i contenuti sono stati privati dei colori; è stato rimosso il menù; nella pagina delle realizzazioni è stata messa in risalto l'immagine selezionata; nella pagina relativa alla vendite si è fatto in modo da non spezzare su più pagine le informazioni relative al singolo articolo all'interno del menù.
	}
}
