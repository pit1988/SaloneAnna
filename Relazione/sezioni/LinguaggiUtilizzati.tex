\section{Linguaggi Utilizzati}{
	La struttura del sito è stata realizzata utilizzando il linguaggio XHTML 1.0, validato correttamente secondo gli standard del W3C. \\
	\\
	La presentazione è stata costruita in CSS, cercando di utilizzare quanto più possibile CSS2, che non valida per poche proprietà usate per una maggiore accessibilità. Risulta invece valido con CSS3, secondo gli standard W3C.
	\\
	Javascript è stato utilizzato per definire funzioni di utlità alle pagine, creare contenuti dinamici (es. aggiungere campi dati nella form lato amministratore) ed effettuare controlli dinamici sui dati inseriti nelle form (pagine contattaci pannello amministratore), per nascondere e mostrare il pannello d'accesso dell'amministratore del sito e mostrare le foto nella pagina delle realizzazioni.\\
	\\
	La gestione dei dati è stata affidata al linguaggio SQL. Il database è stato caricato in uno spazio riservato tramite ‘‘phpmyadmin’’, un'applicazione che permette di gestire la base di dati direttamente dal browser. Non c'è un file di riferimento tra quelli del sito perché il database è stato creato tramite degli appositi comandi, quindi esso è situato in uno spazio riservato a parte.
	\\
	Per il comportamento, oltre a JS è stato utilizzato il linguaggio PHP. Nello specifico esso è stato usato per gestire le sessioni e le operazioni da effettuare sul database.
	\\
}