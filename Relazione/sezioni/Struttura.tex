\section{Struttura}{
	La struttura del sito è stata divisa secondo le operazioni disponibili, così da semplificarne l'uso ed aiutare l'utente a trovare più facilmente le informazioni.
	\\
	Il sito è stato sviluppato in XHTML 1.0 Strict, supportato con PHP per semplificare la definizione e la stampa di codice comune a più pagine.
	\\
	Di seguito sono elencate le pagine sviluppate, insieme ai link per poterle visualizzare da browser; alcuni file PHP non hanno un link associato perché richiedono dei dati in input per poter funzionare, che non possono essere inviati tramite un semplice link:
	%TODO: aggiornare paragrafo sopra e le descrizioni dei punti sotto
	\begin{itemize}\itemsep1pt
		\item \href{http://tecnologie-web.studenti.math.unipd.it/tecweb/~pgabelli/}{public\_html/index.php}: questa pagina rappresenta la ‘‘copertina’’ del sito, il cui scopo è di attirare il cliente;
		\item \href{http://tecnologie-web.studenti.math.unipd.it/tecweb/~pgabelli/public\_html/chi\_siamo.php}{public\_html/chi\_siamo.php}: in questa pagina sono state inserite le informazioni aziendali ed una breve descrizione dell'attività;
		\item \href{http://tecnologie-web.studenti.math.unipd.it/tecweb/~pgabelli/public\_html/foto.php}{public\_html/foto.php}: in questa pagina si trova una galleria fotografica di alcune lavori eseguiti dal Salone;
		\item \href{http://tecnologie-web.studenti.math.unipd.it/tecweb/~pgabelli/public\_html/listino.php}{public\_html/listino.php}: in questa pagina si trova un elenco dei possibili tipi di intervento che l'utente può richiedere, insieme al relativo costo;
		\item \href{http://tecnologie-web.studenti.math.unipd.it/tecweb/~pgabelli/public\_html/contattaci.php}{public\_html/contattaci.php}: qui dentro si trovano una form da cui contattare l'azienda, i numeri di telefono ed una mappa con le indicazioni per raggiungere il Salone;
		\item \href{http://tecnologie-web.studenti.math.unipd.it/tecweb/~pgabelli/public\_html/login.php}{public\_html/login.php}: in questo file si trovano il codice che controlla ed effettua il login e la parte di codice HTML che stampa il successo o meno dell'operazione;
		\item \href{http://tecnologie-web.studenti.math.unipd.it/tecweb/~pgabelli/public\_html/Immagini.php}{public\_html/Immagini.php}: in questa pagina si trova un menù da cui l'utente può accedere alle pagine per aggiungere, eliminare, modificare o visualizzare le immagini di \href{http://tecnologie-web.studenti.math.unipd.it/tecweb/~pgabelli/public\_html/foto.php}{public\_html/foto.php};
		\item \href{http://tecnologie-web.studenti.math.unipd.it/tecweb/~pgabelli/public\_html/NuovaFoto.php}{public\_html/NuovaFoto.php}: in questa pagina l'amministratore può inserire una nuova foto per la galleria, insieme alla sua descrizione;
		\item \href{http://tecnologie-web.studenti.math.unipd.it/tecweb/~pgabelli/public\_html/EliminaFoto.php}{public\_html/EliminaFoto.php}: in questa pagina l'amministratore può eliminare una o più foto della galleria;
		\item \href{http://tecnologie-web.studenti.math.unipd.it/tecweb/~pgabelli/public\_html/SelezionaFoto.php}{public\_html/SelezionaFoto.php}: in questa pagina l'amministratore può selezionare una foto da modificare tra quelle già presenti nella galleria;
		\item \href{http://tecnologie-web.studenti.math.unipd.it/tecweb/~pgabelli/public\_html/ModificaFoto.php}{public\_html/ModificaFoto.php}: in questa pagina l'amministratore può modificare i dati della foto selezionata nella pagina precedente, ovviamente richiede che le sia trasmesso il codice dell'immagine da cambiare;
		\item \href{http://tecnologie-web.studenti.math.unipd.it/tecweb/~pgabelli/public\_html/Prodotti.php}{public\_html/Prodotti.php}: in questa pagina è presente un menù da cui l'amministratore può scegliere di eseguire varie operazioni sui prodotti;
		\item \href{http://tecnologie-web.studenti.math.unipd.it/tecweb/~pgabelli/public\_html/ProdottiQuery.php}{public\_html/ProdottiQuery.php}: in questa pagina sono elencati i prodotti prossimi ad esaurirsi, insieme alla rispettiva quantità presente in magazzino;
		\item \href{http://tecnologie-web.studenti.math.unipd.it/tecweb/~pgabelli/public\_html/ProdottiMax.php}{public\_html/ProdottiMax.php}: in questa pagina sono elencati i prodotti più usati durante gli appuntamenti;
		\item \href{http://tecnologie-web.studenti.math.unipd.it/tecweb/~pgabelli/public\_html/ProdottiClienteAppuntamento.php}{public\_html/ProdottiClienteAppuntamento.php}: in questa pagina l'amministratore può inserire i dati di un cliente per poter aggiungere i prodotti usati in uno dei suoi appuntamenti;
		\item \href{http://tecnologie-web.studenti.math.unipd.it/tecweb/~pgabelli/public\_html/SelezionaAppuntamentoCliente.php}{public\_html/SelezionaAppuntamentoCliente.php}: in questa pagina l'amministratore può selezionare uno degli appuntamenti del cliente che ha scelto nella pagina del punto precedente, in modo da poter aggiungere i prodotti che sono stati usati in quell'appuntamento, ovviamente richiede che siano forniti dei dati alla pagina;
		\item \href{http://tecnologie-web.studenti.math.unipd.it/tecweb/~pgabelli/public\_html/SelezionaProdottiAppuntamento.php}{public\_html/SelezionaProdottiAppuntamento.php}: in questa pagina l'amministratore può completare l'iter, iniziato con i due punti precedenti, inserendo la quantità di ogni prodotto che è stato usato durante l'appuntamento scelto, ovviamente richiede che siano forniti dei dati alla pagina;
		\item \href{http://tecnologie-web.studenti.math.unipd.it/tecweb/~pgabelli/public\_html/EliminaProdotti.php}{public\_html/EliminaProdotti.php}: in questa pagina l'amministratore può eliminare uno o più prodotti presenti nell'inventario;
		\item \href{http://tecnologie-web.studenti.math.unipd.it/tecweb/~pgabelli/public\_html/NuovoProdotto.php}{public\_html/NuovoProdotto.php}: in questa pagina l'amministratore può aggiungere un nuovo prodotto all'inventario inserendone i dati;
		\item \href{http://tecnologie-web.studenti.math.unipd.it/tecweb/~pgabelli/public\_html/StoricoProd.php}{public\_html/StoricoProd.php}: in questa pagina l'amministratore può inserire il nome e il cognome di un utente per visualizzare i prodotti da lui usati;
		\item \href{http://tecnologie-web.studenti.math.unipd.it/tecweb/~pgabelli/public\_html/Inventario.php}{public\_html/Inventario.php}: in questa pagina l'amministratore può visualizzare tutti i dati dei prodotti dell'inventario e modificarne le quantità presenti;
		\item \href{http://tecnologie-web.studenti.math.unipd.it/tecweb/~pgabelli/public\_html/Clienti.php}{public\_html/Clienti.php}: in questa pagina è presente un menù da cui l'amministratore può eseguire delle operazioni sui clienti salvati nel database;
		\item \href{http://tecnologie-web.studenti.math.unipd.it/tecweb/~pgabelli/public\_html/QueryCompleanno.php}{public\_html/QueryCompleanno.php}: in questa pagina sono elencati tutti gli utenti che compiono gli anni entro un mese;
		\item \href{http://tecnologie-web.studenti.math.unipd.it/tecweb/~pgabelli/public\_html/ElencoClienti.php}{public\_html/ElencoClienti.php}: in questa pagina sono elencati tutti i dati relativi ai clienti salvati nel database;
		\item \href{http://tecnologie-web.studenti.math.unipd.it/tecweb/~pgabelli/public\_html/NuovoCliente.php}{public\_html/NuovoCliente.php}: in questa pagina l'amministratore può inserire i dati per aggiungere un nuovo cliente;
		\item \href{http://tecnologie-web.studenti.math.unipd.it/tecweb/~pgabelli/public\_html/ScegliCliente.php}{public\_html/ScegliCliente.php}: in questa pagina l'amministratore può selezionare un cliente per modificarne i dati;
		\item \href{http://tecnologie-web.studenti.math.unipd.it/tecweb/~pgabelli/public\_html/ModificaCliente.php}{public\_html/ModificaCliente.php}: in questa pagina l'amministratore può inserire i dati del cliente scelto nel punto precedente per aggiornarli, ovviamente questa pagina richiede che le siano forniti dei dati;
		\item \href{http://tecnologie-web.studenti.math.unipd.it/tecweb/~pgabelli/public\_html/ConfermaModificaCliente.php}{public\_html/ConfermaModificaCliente.php}: in questa pagina viene confermato all'amministratore che i dati del cliente sono stati aggiornati con quelli che ha inserito nei due punti precedenti, ovviamente richiede che le siano forniti dei dati;
		\item \href{http://tecnologie-web.studenti.math.unipd.it/tecweb/~pgabelli/public\_html/EliminaCliente.php}{public\_html/EliminaCliente.php}: in questa pagina l'amministratore può eliminare uno o più clienti;
		\item \href{http://tecnologie-web.studenti.math.unipd.it/tecweb/~pgabelli/public\_html/Appuntamenti.php}{public\_html/Appuntamenti.php}: in questa pagina è presente un menù da cui l'utente può scegliere di eseguire alcune operazioni sugli appuntamenti salvati nel database;
		\item \href{http://tecnologie-web.studenti.math.unipd.it/tecweb/~pgabelli/public\_html/Toptype.php}{public\_html/Toptype.php}: in questa pagina è presente una tabella con la frequenza dei tipi di appuntamento scelti dai clienti;
		\item \href{http://tecnologie-web.studenti.math.unipd.it/tecweb/~pgabelli/public\_html/AppuntamentiSettimana.php}{public\_html/AppuntamentiSettimana.php}: in questa pagina l'amministratore può visualizzare la lista degli appuntamenti divisi per giorno;
		\item \href{http://tecnologie-web.studenti.math.unipd.it/tecweb/~pgabelli/public\_html/RicercaAppuntamenti.php}{public\_html/RicercaAppuntamenti.php}: in questa pagina l'amministratore può inserire i dati relativi ad un cliente o ad un orario per visualizzare gli appuntamenti di quel cliente o prenotato a quell'orario;
		\item \href{http://tecnologie-web.studenti.math.unipd.it/tecweb/~pgabelli/public\_html/AppClienteGiorno.php}{public\_html/AppClienteGiorno.php}: in questa pagina l'amministratore può visualizzare gli appuntamenti selezionati in base ai dati inseriti nel punto precedente, ovviamente questa pagina richiede che le siano forniti dei dati;
		\item \href{http://tecnologie-web.studenti.math.unipd.it/tecweb/~pgabelli/public\_html/NuovoAppuntamento.php}{public\_html/NuovoAppuntamento.php}: in questa pagina l'amministratore può inserire i dati di un nuovo appuntamento per aggiungerli al database;
		\item \href{http://tecnologie-web.studenti.math.unipd.it/tecweb/~pgabelli/public\_html/ConfermaNuovoAppuntamento.php}{public\_html/ConfermaNuovoAppuntamento.php}: in questa pagina viene confermato all'utente che l'aggiunta dell'appuntamento con i dati inseriti nel punto precedente è avvenuta con successo, ovviamente richiede che le siano forniti dei dati;
		\item \href{http://tecnologie-web.studenti.math.unipd.it/tecweb/~pgabelli/public\_html/ScegliAppuntamento.php}{public\_html/ScegliAppuntamento.php}: in questa pagina l'amministratore può selezionare l'appuntamento a cui vuole modificare i dati;
		\item \href{http://tecnologie-web.studenti.math.unipd.it/tecweb/~pgabelli/public\_html/ModificaAppuntamento.php}{public\_html/ModificaAppuntamento.php}: in questa pagina l'amministratore può inserire i nuovi dati dell'appuntamento scelto nel punto precedente, ovviamente questa pagina richiede che le siano forniti dei dati;
		\item \href{http://tecnologie-web.studenti.math.unipd.it/tecweb/~pgabelli/public\_html/ConfermaModificaAppuntamento.php}{public\_html/ConfermaModificaAppuntamento.php}: in questa pagina viene confermato all'utente che la modifica dell'appuntamento con i dati inseriti nei due punti precedenti è avvenuta con successo, ovviamente richiede che le siano forniti dei dati;
		\item \href{http://tecnologie-web.studenti.math.unipd.it/tecweb/~pgabelli/public\_html/EliminaAppuntamenti.php}{public\_html/EliminaAppuntamenti.php}: in questa pagina l'amministratore può eliminare uno o più appuntamenti tra quelli salvati nel database;
		\item \href{http://tecnologie-web.studenti.math.unipd.it/tecweb/~pgabelli/public\_html/Utilita.php}{public\_html/Utilita.php}: in questa pagina è presente un menù in cui l'amministratore può scegliere di eseguire delle operazioni che non sono presenti nei menù precedenti;
		\item \href{http://tecnologie-web.studenti.math.unipd.it/tecweb/~pgabelli/public\_html/Messaggi.php}{public\_html/Messaggi.php}: in questa pagina l'amministratore può visualizzare tutti i messaggi che sono stati ricevuti;
		\item \href{http://tecnologie-web.studenti.math.unipd.it/tecweb/~pgabelli/public\_html/MostraMessaggio.php}{public\_html/MostraMessaggio.php}: questa pagina viene aperta quando l'amministratore clicca sul contenuto di uno dei messaggi, e qui egli può visualizzare tutti i dati relativi al messaggio, compreso il contenuto integrale, dato che nella pagina precedente esso veniva troncato ad un certo punto, ovviamente questa pagina richiede che le siano forniti dei dati;
		\item \href{http://tecnologie-web.studenti.math.unipd.it/tecweb/~pgabelli/public\_html/EliminaMessaggi.php}{public\_html/EliminaMessaggi.php}: in questa pagina l'amministratore può eliminare uno o più messaggi tra quelli salvati nel database;
		\item \href{http://tecnologie-web.studenti.math.unipd.it/tecweb/~pgabelli/public\_html/CambioPassword.php}{public\_html/CambioPassword.php}: in questa pagina l'amministratore può cambiare la password del proprio account;
		\item \href{http://tecnologie-web.studenti.math.unipd.it/tecweb/~pgabelli/public\_html/errore.php}{public\_html/errore.php}: questa pagina segnala all'utente che si è verificato un errore abbastanza grave, come l'impossibilità di accedere al database;
	\end{itemize}
}