\section{Struttura}{
	La struttura del sito è stata divisa secondo le operazioni disponibili, così da semplificarne l'uso ed aiutare l'utente a trovare più facilmente le informazioni.
	\\
	Il sito è stato sviluppato in XHTML 1.0 Strict; la pagina di ricerca e quella delle vendite sono state generate tramite XSLT, modificando eventualmente il foglio di stile con i file PERL; tramite questi ultimi infine vengono mostrati i messaggi di errore in caso di dati inviati errati o di malfunzionamenti del server o del codice, le varie form di inserimento o modifica dei prodotti e l'esecuzione di queste operazioni in base ai dati inseriti dall'amministratore.
	\\
	Di seguito sono elencate le pagine sviluppate, insieme ai link per poterle visualizzare da browser; alcuni file CGI non hanno un link associato perché richiedono dei dati in input per poter funzionare, che non possono essere inviati tramite un semplice link:
	\begin{itemize}\itemsep1pt
		\item \href{http://tecnologie-web.studenti.math.unipd.it/tecweb/~pgabelli/}{public\_html/index.html}: in questa pagina sono state inserite le informazioni aziendali ed una breve descrizione dell'attività;
		\item \href{http://tecnologie-web.studenti.math.unipd.it/tecweb/~pgabelli/contattaci.html}{public\_html/contattaci.html}: all'interno della quale si trova una form da cui contattare l'azienda, i numeri di telefono ed una mappa con cui raggiungere \textbf{\ggt};
		\item \href{http://tecnologie-web.studenti.math.unipd.it/tecweb/~pgabelli/realizzazioni.html}{public\_html/realizzazioni.html}: in questa pagina si trova una galleria fotografica di alcune realizzazioni di \textbf{\ggt};
		\item \href{http://tecnologie-web.studenti.math.unipd.it/tecweb/~pgabelli/cgi-bin/checkLog.cgi}{cgi-bin/checkLog.cgi}: pagina generata dinamicamente in cui sono esposti i prodotti in vendita, con relativi dettagli, suddivisi tra \textit{piante} ed \textit{attrezzi};
		\item \textbf{cgi-bin/log.cgi}: questa pagina è generata in modo da ricevere le credenziali per l'accesso dell'amministratore, mostrare eventuali errori se si rivelano non corrette, altrimenti mostrare le operazioni che l'amministratore può fare sulla base di dati: inserimento, modifica e cancellazione prodotti, aggiornamento delle credenziali d'accesso;
		\item \href{http://tecnologie-web.studenti.math.unipd.it/tecweb/~pgabelli/cgi-bin/logout.cgi}{cgi-bin/logout.cgi}: qui viene mostrata la pagina dei prodotti in vendita, prima però vengono eliminati i dati relativi alla sessione, aperta in seguito all'accesso come amministratore;
		\item \textbf{cgi-bin/databaseManager.cgi}: tramite questa pagina, utilizzando il passaggio di parametri, è possibile inserire nuovi prodotti, eliminarne e modificare quelli presenti. La pagina viene generata a partire dal file statico \href{http://tecnologie-web.studenti.math.unipd.it/tecweb/~pgabelli/databaseManager.html}{public\_html/databaseManager.html};
		\item \textbf{cgi-bin/email.cgi}: questa pagina viene raggiunta in seguito all'invio e-mail della pagina contattaci; contiene come parte dinamica il messaggio di successo o di errore dell'invio dell'email; la parte statica è identica alla pagina \href{http://tecnologie-web.studenti.math.unipd.it/tecweb/~pgabelli/contattaci.html}{contattaci.html}.
	\end{itemize}
}