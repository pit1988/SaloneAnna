\section{Struttura}{
	La struttura del sito è stata divisa secondo le operazioni disponibili, così da semplificarne l'uso ed aiutare l'utente a trovare più facilmente le informazioni; inoltre per molte operazioni sono presenti più pagine, perché richiedono più passaggi e ognuno di essi deve stare su una pagina a parte.
	\\
	Il sito è stato sviluppato in XHTML 1.0 Strict, supportato con PHP per semplificare la definizione e la stampa di codice comune a più pagine.
	\\
	Di seguito sono elencate le pagine sviluppate, insieme ai link per poterle visualizzare da browser; alcuni file non hanno un link associato perché richiedono dei dati in input per poter funzionare:
	%TODO: controllare se i link funzionano
	\begin{itemize}\itemsep1pt
		\item \href{http://tecweb2016.studenti.math.unipd.it/smarches/}{index.php}: questa pagina rappresenta la ‘‘copertina’’ del sito, il cui scopo è di attirare il cliente;
		\item \href{http://tecweb2016.studenti.math.unipd.it/smarches/chi\_siamo.php}{chi\_siamo.php}: in questa pagina sono state inserite le informazioni aziendali ed una breve descrizione dell'attività;
		\item \href{http://tecweb2016.studenti.math.unipd.it/smarches/foto.php}{foto.php}: in questa pagina si trova una galleria fotografica di alcune lavori eseguiti dal Salone;
		\item \href{http://tecweb2016.studenti.math.unipd.it/smarches/listino.php}{listino.php}: in questa pagina si trova un elenco dei possibili tipi di intervento che l'utente può richiedere, insieme al relativo costo;
		\item \href{http://tecweb2016.studenti.math.unipd.it/smarches/contattaci.php}{contattaci.php}: qui dentro si trovano una form da cui contattare l'azienda, i numeri di telefono ed una mappa con le indicazioni per raggiungere il Salone;
		\item \href{http://tecweb2016.studenti.math.unipd.it/smarches/login.php}{login.php}: in questo file si trovano il codice che controlla ed effettua il login e la parte di codice HTML che stampa il successo o meno dell'operazione;
		\item \href{http://tecweb2016.studenti.math.unipd.it/smarches/Immagini.php}{Immagini.php}: in questa pagina si trova un menù da cui l'utente può accedere alle pagine per aggiungere, eliminare, modificare o visualizzare le immagini di \href{http://tecweb2016.studenti.math.unipd.it/smarches/foto.php}{foto.php};
		\item \href{http://tecweb2016.studenti.math.unipd.it/smarches/NuovaFoto.php}{NuovaFoto.php}: in questa pagina l'amministratore può inserire una nuova foto per la galleria, insieme alla sua descrizione;
		\item \href{http://tecweb2016.studenti.math.unipd.it/smarches/EliminaFoto.php}{EliminaFoto.php}: in questa pagina l'amministratore può eliminare una o più foto della galleria;
		\item \href{http://tecweb2016.studenti.math.unipd.it/smarches/SelezionaFoto.php}{SelezionaFoto.php}: in questa pagina l'amministratore può selezionare una foto da modificare tra quelle già presenti nella galleria;
		\item ModificaFoto.php: in questa pagina l'amministratore può modificare i dati della foto selezionata nella pagina precedente, ovviamente richiede che le sia trasmesso il codice dell'immagine da cambiare;
		\item \href{http://tecweb2016.studenti.math.unipd.it/smarches/Prodotti.php}{Prodotti.php}: in questa pagina è presente un menù da cui l'amministratore può scegliere di eseguire varie operazioni sui prodotti;
		\item \href{http://tecweb2016.studenti.math.unipd.it/smarches/ProdottiQuery.php}{ProdottiQuery.php}: in questa pagina sono elencati i prodotti prossimi ad esaurirsi, insieme alla rispettiva quantità presente in magazzino;
		\item \href{http://tecweb2016.studenti.math.unipd.it/smarches/ProdottiMax.php}{ProdottiMax.php}: in questa pagina sono elencati i prodotti più usati durante gli appuntamenti;
		\item \href{http://tecweb2016.studenti.math.unipd.it/smarches/ProdottiClienteAppuntamento.php}{ProdottiClienteAppuntamento.php}: in questa pagina l'amministratore può inserire i dati di un cliente per poter aggiungere i prodotti usati in uno dei suoi appuntamenti;
		\item SelezionaAppuntamentoCliente.php: in questa pagina l'amministratore può selezionare uno degli appuntamenti del cliente che ha scelto nella pagina del punto precedente, in modo da poter aggiungere i prodotti che sono stati usati in quell'appuntamento, ovviamente richiede che siano forniti dei dati alla pagina;
		\item SelezionaProdottiAppuntamento.php: in questa pagina l'amministratore può completare l'iter, iniziato con i due punti precedenti, inserendo la quantità di ogni prodotto che è stato usato durante l'appuntamento scelto, ovviamente richiede che siano forniti dei dati alla pagina;
		\item \href{http://tecweb2016.studenti.math.unipd.it/smarches/EliminaProdotti.php}{EliminaProdotti.php}: in questa pagina l'amministratore può eliminare uno o più prodotti presenti nell'inventario;
		\item \href{http://tecweb2016.studenti.math.unipd.it/smarches/NuovoProdotto.php}{NuovoProdotto.php}: in questa pagina l'amministratore può aggiungere un nuovo prodotto all'inventario inserendone i dati;
		\item \href{http://tecweb2016.studenti.math.unipd.it/smarches/StoricoProd.php}{StoricoProd.php}: in questa pagina l'amministratore può inserire il nome e il cognome di un utente per visualizzare i prodotti da lui usati;
		\item \href{http://tecweb2016.studenti.math.unipd.it/smarches/Inventario.php}{Inventario.php}: in questa pagina l'amministratore può visualizzare tutti i dati dei prodotti dell'inventario e modificarne le quantità presenti;
		\item \href{http://tecweb2016.studenti.math.unipd.it/smarches/Clienti.php}{Clienti.php}: in questa pagina è presente un menù da cui l'amministratore può eseguire delle operazioni sui clienti salvati nel database;
		\item \href{http://tecweb2016.studenti.math.unipd.it/smarches/QueryCompleanno.php}{QueryCompleanno.php}: in questa pagina sono elencati tutti gli utenti che compiono gli anni entro un mese;
		\item \href{http://tecweb2016.studenti.math.unipd.it/smarches/ElencoClienti.php}{ElencoClienti.php}: in questa pagina sono elencati tutti i dati relativi ai clienti salvati nel database;
		\item \href{http://tecweb2016.studenti.math.unipd.it/smarches/NuovoCliente.php}{NuovoCliente.php}: in questa pagina l'amministratore può inserire i dati per aggiungere un nuovo cliente;
		\item \href{http://tecweb2016.studenti.math.unipd.it/smarches/ScegliCliente.php}{ScegliCliente.php}: in questa pagina l'amministratore può selezionare un cliente per modificarne i dati;
		\item ModificaCliente.php: in questa pagina l'amministratore può inserire i dati del cliente scelto nel punto precedente per aggiornarli, ovviamente questa pagina richiede che le siano forniti dei dati;
		\item ConfermaModificaCliente.php: in questa pagina viene confermato all'amministratore che i dati del cliente sono stati aggiornati con quelli che ha inserito nei due punti precedenti, ovviamente richiede che le siano forniti dei dati;
		\item \href{http://tecweb2016.studenti.math.unipd.it/smarches/EliminaCliente.php}{EliminaCliente.php}: in questa pagina l'amministratore può eliminare uno o più clienti;
		\item \href{http://tecweb2016.studenti.math.unipd.it/smarches/Appuntamenti.php}{Appuntamenti.php}: in questa pagina è presente un menù da cui l'utente può scegliere di eseguire alcune operazioni sugli appuntamenti salvati nel database;
		\item \href{http://tecweb2016.studenti.math.unipd.it/smarches/Toptype.php}{Toptype.php}: in questa pagina è presente una tabella con la frequenza dei tipi di appuntamento scelti dai clienti;
		\item \href{http://tecweb2016.studenti.math.unipd.it/smarches/AppuntamentiSettimana.php}{AppuntamentiSettimana.php}: in questa pagina l'amministratore può visualizzare la lista degli appuntamenti divisi per giorno;
		\item \href{http://tecweb2016.studenti.math.unipd.it/smarches/RicercaAppuntamenti.php}{RicercaAppuntamenti.php}: in questa pagina l'amministratore può inserire i dati relativi ad un cliente o ad un orario per visualizzare gli appuntamenti di quel cliente o prenotato a quell'orario;
		\item AppClienteGiorno.php: in questa pagina l'amministratore può visualizzare gli appuntamenti selezionati in base ai dati inseriti nel punto precedente, ovviamente questa pagina richiede che le siano forniti dei dati;
		\item \href{http://tecweb2016.studenti.math.unipd.it/smarches/NuovoAppuntamento.php}{NuovoAppuntamento.php}: in questa pagina l'amministratore può inserire i dati di un nuovo appuntamento per aggiungerli al database;
		\item ConfermaNuovoAppuntamento.php: in questa pagina viene confermato all'utente che l'aggiunta dell'appuntamento con i dati inseriti nel punto precedente è avvenuta con successo, ovviamente richiede che le siano forniti dei dati;
		\item \href{http://tecweb2016.studenti.math.unipd.it/smarches/ScegliAppuntamento.php}{ScegliAppuntamento.php}: in questa pagina l'amministratore può selezionare l'appuntamento a cui vuole modificare i dati;
		\item ModificaAppuntamento.php: in questa pagina l'amministratore può inserire i nuovi dati dell'appuntamento scelto nel punto precedente, ovviamente questa pagina richiede che le siano forniti dei dati;
		\item ConfermaModificaAppuntamento.php: in questa pagina viene confermato all'utente che la modifica dell'appuntamento con i dati inseriti nei due punti precedenti è avvenuta con successo, ovviamente richiede che le siano forniti dei dati;
		\item \href{http://tecweb2016.studenti.math.unipd.it/smarches/EliminaAppuntamenti.php}{EliminaAppuntamenti.php}: in questa pagina l'amministratore può eliminare uno o più appuntamenti tra quelli salvati nel database;
		\item \href{http://tecweb2016.studenti.math.unipd.it/smarches/Utilita.php}{Utilita.php}: in questa pagina è presente un menù in cui l'amministratore può scegliere di eseguire delle operazioni che non sono presenti nei menù precedenti;
		\item \href{http://tecweb2016.studenti.math.unipd.it/smarches/Messaggi.php}{Messaggi.php}: in questa pagina l'amministratore può visualizzare tutti i messaggi che sono stati ricevuti;
		\item MostraMessaggio.php: questa pagina viene aperta quando l'amministratore clicca sul contenuto di uno dei messaggi, e qui egli può visualizzare tutti i dati relativi al messaggio, compreso il contenuto integrale, dato che nella pagina precedente esso veniva troncato ad un certo punto, ovviamente questa pagina richiede che le siano forniti dei dati;
		\item \href{http://tecweb2016.studenti.math.unipd.it/smarches/EliminaMessaggi.php}{EliminaMessaggi.php}: in questa pagina l'amministratore può eliminare uno o più messaggi tra quelli salvati nel database;
		\item \href{http://tecweb2016.studenti.math.unipd.it/smarches/CambioPassword.php}{CambioPassword.php}: in questa pagina l'amministratore può cambiare la password del proprio account;
		\item errore.php: questa pagina segnala all'utente che si è verificato un errore abbastanza grave, come l'impossibilità di accedere al database, richiede che le sia fornito il codice dell'errore;
	\end{itemize}
}