\section{Presentazione}{
	Per presentare al meglio le informazioni disponibili abbiamo posto la nostra attenzione sulla precisione e l'accessibilità.\\
	Avendo separato contenuto, presentazione e struttura, l'uso del codice CSS ha permesso di curare l'aspetto delle pagine; abbiamo usato CSS versione 3, compatibile con la maggior parte dei browser in uso attualmente.\\
	Per rendere migliore la presentazione, abbiamo suddiviso i file CSS in base alle loro funzioni, arrivando ad avere 4 differenti fogli di stile:
	\begin{itemize}
		\item \textbf{home.css}, utilizzato per la maggior parte dei dispositivi e browser con risoluzione maggiore, quali computer portatili e fissi;
		\item \textbf{print.css}, destinato a semplificare la stampa delle pagine.
		Giustifica e modifica il testo, ingrandendo e cambiando il tipo di carattere in uno di più semplice lettura; porta le immagini al centro della pagina. Toglie infine gli sfondi decorativi e sezioni inutilizzabili su carta per ottenere una stampa più chiara.
		\item \textbf{small-devices.css}: viene usato per i dispositivi mobili quali telefoni e tablet che non offrono schermi ampi e richiedono una visualizzazione chiara dell'informazione. Questo foglio di stile viene attivato a risoluzioni inferiori ai 640px. Rende visibile inoltre parti nascoste come link per tornare a inizio pagina in quanto più difficile lo scorrimento su mobile.
		\item \textbf{explorer.css}: quest'ultimo foglio è destinato ad \textit{Internet Explorer}, in versione 8 od inferiore; rispetto al foglio di stile home si differenzia per un posizionamento differente di alcuni elementi e per l'aver sostituito gli attributi non compatibili.
	\end{itemize}
	Non abbiamo usato font particolari, in questo modo le pagine usano quelli di sistema, garantendo la scalabilità della pagina. I caratteri utilizzati hanno una dimensione espressa in ‘‘em’’ al fine di renderli più adattabili alle preferenze dell'utente senza peggiorare l'aspetto del sito.
	I contenuti del sito si presentano accessibili ed il sito si presenta utilizzabile anche tenendo i fogli di stile disattivati.
	Nelle sezioni in cui il layout è a più colonne si è scelto di utilizzare percentuali ( \% ), questo per avere un sito più flessibile. Tenendo in considerazione problematiche di auto-posizionamento che alcuni browser subiscono.
}
