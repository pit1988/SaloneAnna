\section{Presentazione}{
	Per presentare al meglio le informazioni disponibili abbiamo posto la nostra attenzione sulla precisione e l'accessibilità.\\
	Avendo separato contenuto, presentazione e struttura, l'uso del codice CSS ha permesso di curare l'aspetto delle pagine; abbiamo usato per la maggior parte CSS versione 2 e solo alcuni elementi di CSS versione 3 compatibile comunque con la maggior parte dei browser in uso attualmente.\\
	Per rendere migliore la presentazione, abbiamo suddiviso i file CSS in base alle loro funzioni, arrivando ad avere 4 differenti fogli di stile:
	\begin{itemize}
		\item \textbf{home.css}, utilizzato per la maggior parte dei dispositivi e browser con risoluzione maggiore, quali computer portatili e fissi;
		\item \textbf{print.css}, destinato a semplificare la stampa delle pagine.
		Giustifica e modifica il testo, ingrandendo e cambiando il tipo di carattere in uno di più semplice lettura; porta le immagini al centro della pagina. Toglie infine gli sfondi decorativi e sezioni inutilizzabili su carta per ottenere una stampa più chiara.
		\item \textbf{small-devices.css}: viene usato per i dispositivi mobili quali telefoni e tablet che non offrono schermi ampi e richiedono una visualizzazione chiara dell'informazione. Questo foglio di stile viene attivato a risoluzioni inferiori ai 640px. Particolare attenzione si è posta sulle tabelle, infatti la versione mobile, grazie all'uso di comandi CSS versione 3, abbiamo dato scelto una politica reattiva senza l'uso di Javascritp e ad usabilità massima. 
		\item \textbf{explorer.css}: quest'ultimo foglio è destinato ad \textit{Internet Explorer}, in versione 8 od inferiore; rispetto al foglio di stile home si differenzia per un posizionamento differente di alcuni elementi e per l'aver sostituito gli attributi non compatibili.
	\end{itemize}
	Abbiamo usato font particolari, ma mantenendo come seconda scelta prioritaria quelli di sistema, solo nei titoli principali. I caratteri utilizzati hanno una dimensione espressa in ‘‘em’’ al fine di renderli più adattabili alle preferenze dell'utente senza peggiorare l'aspetto del sito.
	I contenuti del sito si presentano accessibili ed il sito si presenta utilizzabile anche tenendo i fogli di stile disattivati.
	Nelle sezioni in cui il layout è a più colonne si è scelto di utilizzare percentuali ( \% ), questo per avere un sito più flessibile. Tenendo in considerazione problematiche di auto-posizionamento che alcuni browser subiscono.\\
	Il sito può essere tracciato sulla base del numero di colonne e della loro disposizione ed è stato utilizzato layout a singola colonna (o layout monolitico). I contenuti della pagina sono distribuiti all'interno di un'unica area in quanto non è prevista una sidebar.\\
	Mentre lo sviluppo orizzontale si è scelto ul layout fluido, questo effetto si è ottenuto impostando dei valori percentuali per la larghezza del contenitore oppure applicando un semplice padding al corpo della pagina.\\
	Capacità di adattamento, cosiddetti layout responsivi (o adattivi). Layout in grado di adattarsi automaticamente alle caratteristiche del display in uso mediante l'utilizzo di vari CSS. L'utilizzo di questa tecnica (responsive design) consente di realizzare un'unica versione del sito web che potrà essere visualizzata efficacemente (e con risultati ottimali) sia sui classici computer che sui moderni device mobili (come smartphone e tablet).
}
