
\rhead{Introduzione}
\section{Introduzione}
\subsection{Descrizione generale}
Il progetto SaloneAnna è strutturato su una base di dati mysql per la gestione di un negozio di parrucchieri. Fulcro del programma risiede nell'amministrazione dei clienti, salvati nel database e collegati agli appuntamenti gestiti dall'amministratore.\\
Il contesto su cui si appoggia la scelta di questo programma riguarda un salone di piccole dimensioni, si parla di un prodotto accessibile da tutte le dipendenti e titolari per tenere aggiornato il negozio con la massima dell'efficienza.\\
Sempre da parte dell'amministratore del salone e chi ha l'accesso al sito ha la possibilità di gestire un magazzino, cancellando, inserendo o modificando ogni singolo prodotto per avere un inventario in evoluzione e ben gestibile. Un utente non registrato invece potrà visualizzare le informazioni generiche del negozio, sfogliare una galleria dinamica delle realizzazioni e contattare tramite form il negozio.\\
\subsection{Caratteristiche degli utenti}{
	Gli utenti del sito saranno persone in ricerca di nuovo negozio nella zona, in grado di utilizzare strumenti per la navigazione web o alle prime armi. \\
	Il sito è rivolto ad un pubblico generico, all'interno del quale possiamo individuare le seguenti categorie:
	\begin{description}\itemsep1pt
		\item[Categoria di utenti:] privati;
		\begin{description}\itemsep1pt
			\item[Funzionalità:] Informarsi sulla locazione, il numero da contattare. Consultare realizzazioni e contattare il negozio lasciando il proprio indirizzo mail.
			\item[Termini generali:] Non eccessivamente distante dal punto vendita, in un raggio di circa 70 Km.
		\end{description}
		\item[Categoria di utenti:] amministratori
		\begin{description}\itemsep1pt
			\item[Funzionalità:] area riservata da cui aggiungere, rimuovere o aggiornare prodotti. Inserire nuovi clienti e conseguentemente gestire appuntamenti. 
		\end{description}
	\end{description}
	\subsection{Vincoli generali}{
		\begin{itemize}\itemsep1pt
			\item Il sito dev'essere accessibile da parte di categorie d'utenti diversificate ed utilizzando dispositivi diversi compresi smartphones e tablet;
			\item Il sito deve presentare possibilità di stampa flessibile a seconda della pagina richiesta;
			\item Il sito dev'essere visitabile tramite i seguenti browser: 
			\begin{itemize}
				\item Firefox 3.6;
				\item Internet Explorer dalla versione 7 alla versione 11; Edge 13;
				\item Chrome 14;
				\item Opera 12.16;
				\item Safari 9.
			\end{itemize}
			\item Separazione tra struttura, presentazione, comportamento;
			\item Conformità agli standard W3C per XHTML, CSS, JS;
			\item Sito comprensibile da screen-reader.
		\end{itemize}
	}
	\subsection{Requisiti}{
		Di seguito sono presentati i requisiti emersi dall'analisi iniziale e quelli che si sono aggiunti nel corso dello svolgimento del progetto. Ciascuno è identificato da un numero progressivo per semplificarne l'individuazione successiva.\\
		\newcounter{magicrownumbers}
		\newcommand\rownumber{\stepcounter{magicrownumbers}\arabic{magicrownumbers}}
		\begin{table}[h]
			\centering
			\begin{tabular}{|p{\dimexpr 0.15\linewidth-2\tabcolsep}|p{\dimexpr 0.8\linewidth-2\tabcolsep}|}
				\hline
				\textbf{ID Req.} & \textbf{Descrizione}\\
				\hline
				\centering \rownumber	&	Il sito dev'essere visualizzabile sui browser elencati all'interno di "Vincoli generali"\\
				\hline
				\centering \rownumber	&	Il sito dev'essere accessibile indipendentemente dalla grandezza dello schermo del dispositivo\\
				\hline
				\centering \rownumber	&	Il sito dev'essere fruibile anche senza richiedere un foglio di stile\\
				\hline
				\centering \rownumber	&	Le figure significative dovranno essere comprensive di un attributo alt per favorire l'accesso ad utenti non vedenti\\
				\hline
				\centering \rownumber	&	Ad i tag quali <input> e <textarea> devono essere associati tabindex e accesskey\\
				\hline
				\centering \rownumber	&	Le gradazioni di colori non devono risultare sgradevoli o di intralcio a persone affette da daltonismo\\
				\hline
				\centering \rownumber	&	Il layout deve risultare fluido nel ridimensionamento del carattere tramite i tasti Ctrl + e Ctrl -\\
				\hline
				\centering \rownumber	&	Il sito deve essere validato per la parte di XHTML2.0, CSS3 e secondo gli standard WAI\\
				\hline
			\end{tabular}
			\label{tab:requisiti}
			\caption{Elenco dei requisiti}
		\end{table}
	}
    \subsection{Riferimenti}{
		Per la progettazione del sito, si è fatto riferimento alle seguenti normative e specifiche:
		\begin{itemize}\itemsep0.5pt
%			\item Legge Stanca \url{www.agid.gov.it/agenda-digitale/pubblica-amministrazione/accessibilita};
			\item Legislazione e linee guida per accessibilità dei siti web istituzionali, 2011 \url{http://www.math.unipd.it/~artico/direttiva.htm}
			\item Specifiche Web Accessibility Initiative (WAI)  \url{http://www.w3.org/WAI};
			\item Specifiche Web Content Accessibility Guidelines (WCAG) 2.0 \url{www.w3.org/TR/WCAG20/};
			\item Specifiche Sezione 508 \url{https://www.section508.gov/};
			\item Ruota dei colori accessibile \url{http://colorfilter.wickline.org/};
			\item Slides del corso: \url{http://docenti.math.unipd.it/gaggi/tecweb/materiale.html}.
			\item Risorse di WebAIM, Web Accessibility In Mind: \url{http://webaim.org/resources}
		\end{itemize}
    }
}
